\documentclass[UTF8]{ctexart}

\usepackage{geometry}
\geometry{a4paper, left=2.5cm, right=2.5cm, top=2.5cm, bottom=2.5cm}
\usepackage{graphicx}
\usepackage{amsmath}
\usepackage{amssymb}
\usepackage{float}
\usepackage{listings}
\usepackage{xcolor}
\usepackage{hyperref}

\hypersetup{
    colorlinks=true,
    linkcolor=blue,
    filecolor=magenta,      
    urlcolor=cyan,
}

% Code listing style
\definecolor{codegreen}{rgb}{0,0.6,0}
\definecolor{codegray}{rgb}{0.5,0.5,0.5}
\definecolor{codepurple}{rgb}{0.58,0,0.82}
\definecolor{backcolour}{rgb}{0.95,0.95,0.92}

\lstdefinestyle{mystyle}{
    backgroundcolor=\color{backcolour},   
    commentstyle=\color{codegreen},
    keywordstyle=\color{magenta},
    numberstyle=\tiny\color{codegray},
    stringstyle=\color{codepurple},
    basicstyle=\ttfamily\footnotesize,
    breakatwhitespace=false,         
    breaklines=true,                 
    captionpos=b,                    
    keepspaces=true,                 
    numbers=left,                    
    numbersep=5pt,                  
    showspaces=false,                
    showstringspaces=false,
    showtabs=false,                  
    tabsize=2
}

\lstset{style=mystyle}

\title{\bfseries 基于双均线交叉的量化交易策略研究}
\author{Gemini AI}
\date{\today}

\begin{document}

\maketitle

\begin{abstract}
移动平均线(Moving Average, MA)是金融市场技术分析中应用最广泛的指标之一。本文旨在探讨并实现一种经典的量化交易策略——双均线交叉策略。我们首先回顾了移动平均线的理论基础,包括其发明者格兰威尔(Joseph E. Granville)提出的八大法则,以及其有效性的理论支撑——资产价格的均值回归特性。接着,本文详细阐述了双均线策略的逻辑核心,即通过短期均线和长期均线的交叉(“黄金交叉”与“死亡交叉”)来产生交易信号。最后,我们基于Python和`gm.api`量化框架,提供了一个完整的策略实现代码,并以螺纹钢期货(SHFE.rb2101)的分钟级数据进行了回测配置说明。本文为理解和实践双均线策略提供了一个系统性的框架。
\end{abstract}

\section{引言}

移动平均线(Moving Average, MA),作为技术分析领域的基石,是投资者用于判断市场趋势、识别买卖时机的重要工具。它最早由美国投资专家格兰威尔(Joseph E. Granville)在20世纪中期提出,其核心思想是通过计算特定周期内价格的算术平均值,来平滑短期价格波动,揭示市场的主要运行趋势\cite{granville}。

格兰威尔在其著作中提出了著名的“八大法则”,仅利用股价与单条移动平均线的相对位置和运动方向,便构建了一套完整的交易信号体系。其中,当价格由下向上突破均线时形成的“黄金交叉”(Golden Cross)和价格由上向下跌破均线时形成的“死亡交叉”(Death Cross),因其简洁有效,至今仍被广大投资者奉为圭臬。

均线理论的有效性,可以从金融经济学的角度找到支撑。诺贝尔经济学奖得主Robert Shiller(1981)的研究发现,从长期来看,资产价格存在向其内在价值或长期均值回归的倾向\cite{shiller}。移动平均线正是对这种“均值”的近似刻画,从而为趋势跟踪和反转交易提供了理论依据。

然而,传统的单均线策略存在一个显著缺陷——滞后性。由于MA是对历史数据的平均,其发出的交易信号往往晚于最佳的入场或离场点,尤其在震荡行情中容易产生“Whipsaw”(左右挨耳光)现象,导致频繁交易和亏损。为了克服这一缺陷,市场演化出了多种改进方法,其中最著名和最常用的便是本文将要深入探讨的“双均线交叉策略”。该策略通过引入两条不同周期的均线(一条短期,一条长期),利用它们之间的交叉来生成交易信号,以期过滤掉部分市场噪音,提高信号的可靠性。

\section{策略原理与构建}

\subsection{移动平均线}

\subsubsection{定义}
简单移动平均线(Simple Moving Average, SMA)是在特定时间周期 $N$ 内,对资产收盘价 $P$ 进行算术平均。其数学表达式为:
$$
SMA_N(t) = \frac{1}{N} \sum_{i=0}^{N-1} P_{t-i}
$$
其中,$P_{t-i}$ 表示在当前时间点 $t$ 之前的第 $i$ 个周期的收盘价。

\subsubsection{格兰威尔八大法则}
格兰威尔八大法则为移动平均线的应用奠定了基础,它系统性地描述了股价与均线在不同关系下所代表的市场意义。如图\ref{fig:granville}所示,这八大法则定义了四个买入点和四个卖出点。

\begin{figure}[H]
    \centering
    \includegraphics[width=0.8\textwidth]{granville.png}
    \caption{格兰威尔八大法则示意图(图片来源于网络)}
    \label{fig:granville}
\end{figure}

\subsection{双均线交叉策略逻辑}
双均线策略是对格兰威尔法则核心思想的提炼与改进。它不再观察价格与单条均线的关系,而是观察两条不同周期均线的相对关系,通常为一条短周期均线(如20周期)和一条长周期均线(如60周期)。
\begin{itemize}
    \item \textbf{短周期均线}:对近期价格变化反应更灵敏,代表了市场的短期动量。
    \item \textbf{长周期均线}:对价格变化反应较迟钝,代表了市场的长期趋势。
\end{itemize}
该策略的交易信号完全基于这两条均线的交叉行为:
\begin{enumerate}
    \item \textbf{做多信号(黄金交叉)}:当短周期均线从下向上穿越长周期均线时,表明短期动量开始强于长期趋势,市场可能进入上升通道。此时,策略将执行买入开仓操作。
    \item \textbf{做空信号(死亡交叉)}:当短周期均线从上向下穿越长周期均线时,表明短期动量开始弱于长期趋势,市场可能进入下降通道。此时,策略将执行卖出开仓(做空)操作。
\end{enumerate}
为了避免持有多空双向头寸,本策略在每次开新仓前,都会先将已有的反向持仓进行平仓。

\section{策略实现与回测}

本节将展示双均线策略的具体代码实现,并定义回测环境的各项参数。

\subsection{回测设置}
\begin{itemize}
    \item \textbf{交易标的}: 螺纹钢2101合约 (SHFE.rb2101)
    \item \textbf{数据频率}: 60秒 (1分钟) K线
    \item \textbf{回测周期}: 2020-04-01 09:00:00 到 2020-05-31 15:00:00
    \item \textbf{初始资金}: 30,000 人民币
    \item \textbf{短周期均线}: 20周期
    \item \textbf{长周期均线}: 60周期
    \item \textbf{交易逻辑}: 市价单交易。每次开仓前先平掉所有已有持仓。
\end{itemize}

\subsection{Python代码实现}
以下代码是使用`gm.api`和`talib`库实现的完整策略。代码结构清晰,主要包含三个函数:`init()`用于策略初始化,`on_bar()`是策略核心逻辑,在每个新的K线数据到来时触发,`on_order_status()`用于处理委托状态回报,确保平仓成功后再进行开仓,增强了策略的稳健性。

\begin{lstlisting}[language=Python, caption={双均线策略Python实现代码}, label={code:strategy}]
# coding=utf-8
from __future__ import print_function, absolute_import
from gm.api import *
import talib

'''
本策略以SHFE.rb2101为交易标的,根据其一分钟(即60s频度)bar数据建立双均线模型,
短周期为20,长周期为60,当短期均线由上向下穿越长期均线时做空,
当短期均线由下向上穿越长期均线时做多,每次开仓前先平掉所持仓位,再开仓。
注:为了适用于仿真和实盘,在策略中增加了一个“先判断是否平仓成功再开仓”的判断逻辑,
以避免出现未平仓成功,可用资金不足的情况。
回测数据为:SHFE.rb2101的60s频度bar数据
回测时间为:2020-04-01 09:00:00到2020-05-31 15:00:00
'''

def init(context):
    context.short = 20              # 短周期均线
    context.long = 60               # 长周期均线
    context.symbol = 'SHFE.rb2101'  # 订阅交易标的
    context.period = context.long + 1 # 订阅数据滑窗长度
    context.open_long = False       # 开多单标记
    context.open_short = False      # 开空单标记
    subscribe(context.symbol, '60s', count=context.period) # 订阅行情

def on_bar(context, bars):
    # 获取通过subscribe订阅的数据
    prices = context.data(context.symbol, '60s', context.period, fields='close')
    
    # 利用talib库计算长短周期均线
    short_avg = talib.SMA(prices.values.reshape(context.period), context.short)
    long_avg = talib.SMA(prices.values.reshape(context.period), context.long)
    
    # 查询持仓
    position_long = context.account().position(symbol=context.symbol, side=PositionSide_Long)
    position_short = context.account().position(symbol=context.symbol, side=PositionSide_Short)
    
    # 死亡交叉:短均线下穿长均线,做空
    # (即当前时间点短均线处于长均线下方,前一时间点短均线处于长均线上方)
    if long_avg[-2] < short_avg[-2] and long_avg[-1] >= short_avg[-1]:
        # 无多仓情况下,直接开空
        if not position_long:
            order_volume(symbol=context.symbol, volume=1, side=OrderSide_Sell, 
                         position_effect=PositionEffect_Open, order_type=OrderType_Market)
            print(f'{context.now} {context.symbol} 市价开空仓')
        # 有多仓情况下,先平多,再开空(开空命令放在on_order_status里面)
        else:
            context.open_short = True
            order_volume(symbol=context.symbol, volume=position_long.volume, side=OrderSide_Sell, 
                         position_effect=PositionEffect_Close, order_type=OrderType_Market)
            print(f'{context.now} {context.symbol} 市价平多仓,准备开空仓')

    # 黄金交叉:短均线上穿长均线,做多
    # (即当前时间点短均线处于长均线上方,前一时间点短均线处于长均线下方)
    if short_avg[-2] < long_avg[-2] and short_avg[-1] >= long_avg[-1]:
        # 无空仓情况下,直接开多
        if not position_short:
            order_volume(symbol=context.symbol, volume=1, side=OrderSide_Buy, 
                         position_effect=PositionEffect_Open, order_type=OrderType_Market)
            print(f'{context.now} {context.symbol} 市价开多仓')
        # 有空仓的情况下,先平空,再开多(开多命令放在on_order_status里面)
        else:
            context.open_long = True
            order_volume(symbol=context.symbol, volume=position_short.volume, side=OrderSide_Buy,
                         position_effect=PositionEffect_Close, order_type=OrderType_Market)
            print(f'{context.now} {context.symbol} 市价平空仓,准备开多仓')

def on_order_status(context, order):
    # 查看下单后的委托状态 (3:全成, 4:部分成交, 5:已撤, 6:拒单)
    if order.status == OrderStatus_Filled:
        # 如果是平仓委托全成
        if order.position_effect == PositionEffect_Close:
            # 如果之前标记了要开空仓
            if context.open_short and order.side == OrderSide_Sell:
                context.open_short = False
                order_volume(symbol=context.symbol, volume=1, side=OrderSide_Sell, 
                             position_effect=PositionEffect_Open, order_type=OrderType_Market)
                print(f'{context.now} {context.symbol} 平多仓成功,市价开空仓')

            # 如果之前标记了要开多仓
            if context.open_long and order.side == OrderSide_Buy:
                context.open_long = False
                order_volume(symbol=context.symbol, volume=1, side=OrderSide_Buy, 
                             position_effect=PositionEffect_Open, order_type=OrderType_Market)
                print(f'{context.now} {context.symbol} 平空仓成功,市价开多仓')

if __name__ == '__main__':
    run(strategy_id='dual_moving_average_strategy',
        filename='main.py',
        mode=MODE_BACKTEST,
        token='Your_Token_Here',
        backtest_start_time='2020-04-01 09:00:00',
        backtest_end_time='2020-05-31 15:00:00',
        backtest_adjust=ADJUST_NONE,
        backtest_initial_cash=30000,
        backtest_commission_ratio=0.0001,
        backtest_slippage_ratio=0.0001)

\end{lstlisting}

\subsection{代码逻辑解析}
该策略的核心在于`on_bar`函数中的交叉判断逻辑。以死亡交叉为例,条件 `long_avg[-2] < short_avg[-2] and long_avg[-1] >= short_avg[-1]` 精确地捕捉了交叉发生的那个K线。`[-1]`代表当前最新完成的K线周期,而`[-2]`代表上一个周期。该条件意为:在上一个周期,短均线仍在长均线之上;而在当前周期,短均线已经等于或低于长均线。这个精确的定义保证了信号在交叉发生的瞬间被触发,且不会重复触发。

此外,通过`on_order_status`回调函数对平仓委托进行监控,确保了“先平后开”逻辑的严密性,这在实盘交易中对于管理风险和资金至关重要。

\section{结论与展望}

本文详细阐述了一个基于双均线交叉的量化交易策略。从格兰威尔八大法则的历史渊源,到均值回归的理论基础,再到具体的策略逻辑设计和代码实现,我们构建了一个完整的、可执行的交易模型。该策略逻辑清晰、易于理解,是量化交易入门的经典范例。

尽管双均线策略简单有效,但它并非完美无缺。在趋势明显的市场中表现较好,但在长期盘整的市场中,可能会因为频繁的无效交叉而导致亏损。未来的研究可以从以下几个方面对该策略进行优化:
\begin{enumerate}
    \item \textbf{参数优化}: 通过历史数据回测,寻找最优的短、长均线周期组合。
    \item \textbf{增加过滤条件}: 结合其他技术指标(如ADX, RSI)来过滤掉震荡行情中的交易信号。
    \item \textbf{动态周期调整}: 根据市场的波动性(如使用ATR指标)动态调整均线的计算周期。
    \item \textbf{风险管理}: 引入止损(Stop-Loss)和止盈(Take-Profit)机制来控制单笔交易的风险和回报。
\end{enumerate}

通过不断地迭代和优化,这个基础策略可以演化成更加稳健和高效的交易系统。

\begin{thebibliography}{9}
    \bibitem{granville}
    Granville, J. E. (1962). \textit{A Strategy of Daily Stock Market Timing for Maximum Profit}. Prentice-Hall.
    
    \bibitem{shiller}
    Shiller, R. J. (1981). Do stock prices move too much to be justified by subsequent changes in dividends?. \textit{American Economic Review}, 71(3), 421-436.

    \bibitem{baike}
    部分背景资料来源于百度百科关于“格兰威尔八大法则”的词条。
\end{thebibliography}

\end{document}